\usepackage{xcolor}
\usepackage{afterpage}
\usepackage{pifont,mdframed}
\usepackage[bottom]{footmisc}

\makeatletter
\gdef\this@inputfilename{input.txt}
\gdef\this@outputfilename{output.txt}
\makeatother

\newcommand{\inputfile}{\texttt{input.txt}}
\newcommand{\outputfile}{\texttt{output.txt}}

\newenvironment{warning}
  {\par\begin{mdframed}[linewidth=2pt,linecolor=gray]%
    \begin{list}{}{\leftmargin=1cm
                   \labelwidth=\leftmargin}\item[\Large\ding{43}]}
  {\end{list}\end{mdframed}\par}

	Giorgio vuole entrare nella setta \emph{Oli-3}, in cui si studiano numeri occulti e misteriosi con cui dominare il mondo. Per essere ammesso, gli viene richiesto di provare il suo valore individuando il \emph{numero della cabala} $C$. Dopo lunghe e difficoltose ricerche, Giorgio \`e riuscito a scoprire che questo numero:
	\begin{itemize}
		\item ha al pi\`u $N$ cifre;
		\item tutte le sue cifre sono multiple di 3 (ma nessuna è zero);
		\item non vi sono due cifre adiacenti uguali;
		\item il resto di $C$ modulo un certo numero $M$ \`e pi\`u grande possibile.
	\end{itemize}
	Aiuta Giorgio a trovare il numero della Cabala ed entrare quindi nella agognata setta!

\Implementation
Dovrai sottoporre esattamente un file con estensione \texttt{.c}, \texttt{.cpp} o \texttt{.pas}.

\begin{warning}
Tra gli allegati a questo task troverai un template (\texttt{cabala.c}, \texttt{cabala.cpp}, \texttt{cabala.pas}) con un esempio di implementazione da completare.
\end{warning}

Se sceglierai di utilizzare il template, dovrai implementare la seguente funzione:
\begin{center}\begin{tabularx}{\textwidth}{|c|X|}
\hline
C/C++  & \verb|long long occulta(int N, int M);|\\
\hline
Pascal & \verb|function occulta(N, M: longint): int64;|\\
\hline
\end{tabularx}\end{center}
In cui:
\begin{itemize}[nolistsep]
  \item L'intero $N$ rappresenta il numero di cifre di $C$.
  \item L'intero $M$ rappresenta il modulo.
  \item La funzione dovrà restituire il \emph{massimo resto} per un numero $C$ che rispetta i vincoli nel testo modulo $M$, che verrà stampato sul file di output.
\end{itemize}

\InputFile
Il file \inputfile{} è composto da $T+1$ righe. La prima riga contiene l'unico intero $T$, il numero di test a cui il tuo programma dovr\`a rispondere. La successive $T$ righe contengono ciascuna due interi $N$ ed $M$ separati da uno spazio.

\OutputFile
Il file \outputfile{} è composto da un'unica riga contenente $T$ interi, le risposte ai test nell'ordine in cui sono stati presentati.

\pagebreak
% Assunzioni
\Constraints
\begin{itemize}[nolistsep, itemsep=2mm]
	\item $1 \le T \le 50$.
	\item $1 \le N \le 18$ per ogni test.
	\item $2 \le M \le 10^9$ per ogni test.
\end{itemize}

\Scoring
Il tuo programma verrà testato su diversi test case raggruppati in subtask.
Per ottenere il punteggio relativo ad un subtask, è necessario risolvere
correttamente tutti i test relativi ad esso.

\begin{itemize}[nolistsep,itemsep=2mm]
  \item \textbf{\makebox[2cm][l]{Subtask 1} [10 punti]}: Casi d'esempio.
  \item \textbf{\makebox[2cm][l]{Subtask 2} [20 punti]}: $N \leq 5$.
  \item \textbf{\makebox[2cm][l]{Subtask 3} [40 punti]}: $N \leq 11$.
  \item \textbf{\makebox[2cm][l]{Subtask 4} [30 punti]}: Nessuna limitazione specifica.
\end{itemize}

% Esempi


\Examples
\begin{example}
\exmpfile{cabala.input0.txt}{cabala.output0.txt}%
\exmpfile{cabala.input1.txt}{cabala.output1.txt}%
\end{example}


\Explanation
Nel \textbf{primo caso di esempio}, viene richiesto un solo test in cui il numero della cabala \`e 6.\\[2mm]
Nel \textbf{secondo caso di esempio}, nel primo test il numero della cabala \`e 63 perch\'e 66 ha due cifre uguali consecutive, per lo stesso motivo nel secondo test \`e 969, mentre nel terzo test il resto 31 pu\`o essere ottenuto per esempio con il numero della cabala 639.
