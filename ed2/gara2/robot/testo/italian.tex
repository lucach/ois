\usepackage{xcolor}
\usepackage{afterpage}
\usepackage{pifont,mdframed}
\usepackage[bottom]{footmisc}

\makeatletter
\gdef\this@inputfilename{input.txt}
\gdef\this@outputfilename{output.txt}
\makeatother

\newcommand{\inputfile}{\texttt{input.txt}}
\newcommand{\outputfile}{\texttt{output.txt}}

\newenvironment{warning}
  {\par\begin{mdframed}[linewidth=2pt,linecolor=gray]%
    \begin{list}{}{\leftmargin=1cm
                   \labelwidth=\leftmargin}\item[\Large\ding{43}]}
  {\end{list}\end{mdframed}\par}

	William ha finito di costruire un robot camminatore, che si muove su una griglia $N \times M$ leggendo i caratteri scritti su di essa e interpretandoli come comandi. In particolare, reagisce in questo modo:
	\begin{itemize}
		\item \texttt{X}: $\quad$ sta fermo;
		\item \texttt{+}: $\quad$ effettua un passo (e cioè si muove di una casella) nella direzione in cui è orientato;
		\item \texttt{O}: $\quad$ effettua un salto lungo due passi nella direzione in cui è orientato;
		\item \texttt{L}: $\quad$ ruota di $90^\circ$ a sinistra e poi effettua un passo nella direzione in cui ora è orientato;
		\item \texttt{R}: $\quad$ ruota di $90^\circ$ a destra e poi effettua un passo nella direzione in cui ora è orientato.
	\end{itemize}
	Nella griglia non sono presenti altri caratteri, né il robot reagisce ad altri comandi oltre a questi.
	Inoltre, la griglia su cui si muove è \emph{toroidale}: vale a dire, ogni qualvolta il robot uscirebbe dal lato destro rientra da quello sinistro (e viceversa), e ogni qualvolta il robot uscirebbe dal lato superiore rientra da quello inferiore (e viceversa).

	William ha ora intenzione di lasciare il robot nella casella $(0,0)$ orientato verso est, e vedere che percorso seguirà. Tuttavia, è preoccupato dall'eventualità che il percorso possa essere ciclico. Aiutalo verificando se il percorso seguito dal robot è ciclico, e se sì calcolando quanto è lunga la sua parte periodica (cioè che si ripete)!

\Implementation
Dovrai sottoporre esattamente un file con estensione \texttt{.c}, \texttt{.cpp} o \texttt{.pas}.

\begin{warning}
Tra gli allegati a questo task troverai un template (\texttt{robot.c}, \texttt{robot.cpp}, \texttt{robot.pas}) con un esempio di implementazione da completare.
\end{warning}

Se sceglierai di utilizzare il template, dovrai implementare la seguente funzione:
\begin{center}\begin{tabularx}{\textwidth}{|c|X|}
\hline
C/C++  & \verb|int osserva(int N, int M, matrix T);|\\
\hline
Pascal & \verb|function osserva(N, M: longint; var T: matrix): longint;|\\
\hline
\end{tabularx}\end{center}
In cui:
\begin{itemize}[nolistsep]
  \item Gli interi $N$, $M$ rappresentano le due dimensioni della griglia.
  \item La matrice \texttt{T}, indicizzata da $0$ a $N-1$ e da $0$ a $M-1$, contiene i caratteri presenti su ogni quadretto.
  \item La funzione dovrà restituire la lunghezza della parte periodica seguita dal robot (o il valore \texttt{-1} se il percorso non è periodico), che verrà stampata sul file di output.
\end{itemize}

\InputFile
Il file \inputfile{} è composto da $N+1$ righe. La prima riga contiene i due interi $N$ e $M$. Le successive $N$ righe contengono $M$ interi ciascuna. In particolare l'$i$-esima di queste righe contiene \texttt{T[$i$][$j$]} per $j=0 \ldots M-1$ separati da uno spazio.

\OutputFile
Il file \outputfile{} è composto da un'unica riga contenente un unico intero, la risposta a questo problema.

% Assunzioni
\Constraints
\begin{itemize}[nolistsep, itemsep=2mm]
	\item $1 \le N, M \le 500$.
	\item \texttt{T[$i$][$j$]} è un carattere tra \texttt{'X+OLR'}, per ogni $i=0\ldots N-1$, $j=0\ldots M-1$.
	\item La lunghezza della parte periodica va calcolata in numero di passi (e quindi un salto conta 2).
\end{itemize}

\Scoring
Il tuo programma verrà testato su diversi test case raggruppati in subtask.
Per ottenere il punteggio relativo ad un subtask, è necessario risolvere
correttamente tutti i test relativi ad esso.

\begin{itemize}[nolistsep,itemsep=2mm]
  \item \textbf{\makebox[2cm][l]{Subtask 1} [10 punti]}: Casi d'esempio.
  \item \textbf{\makebox[2cm][l]{Subtask 2} [20 punti]}: Sono presenti solo i caratteri \texttt{'X+O'}.
  \item \textbf{\makebox[2cm][l]{Subtask 3} [20 punti]}: Sono presenti solo i caratteri \texttt{'LR'}.
  \item \textbf{\makebox[2cm][l]{Subtask 4} [10 punti]}: Sono presenti solo i caratteri \texttt{'+LR'}.
  \item \textbf{\makebox[2cm][l]{Subtask 5} [20 punti]}: Sono presenti solo i caratteri \texttt{'+OLR'}.
  \item \textbf{\makebox[2cm][l]{Subtask 6} [20 punti]}: Nessuna limitazione specifica.
\end{itemize}

% Esempi


\Examples
\begin{example}
\exmpfile{robot.input0.txt}{robot.output0.txt}%
\exmpfile{robot.input1.txt}{robot.output1.txt}%
\exmpfile{robot.input2.txt}{robot.output2.txt}%
\end{example}


\pagebreak
\Explanation
Nel \textbf{primo caso di esempio}, il percorso seguito è questo (seguendo i colori rosso, blu, verde, porpora, grigio):\\[2mm]
\begin{figure}[H]%
\centering\includegraphics{asy_robot/fig1.pdf}%
\end{figure}
Nel \textbf{secondo caso di esempio}, il percorso seguito è questo (seguendo i colori rosso, blu, verde, porpora, grigio):\\[2mm]
\begin{figure}[H]%
\centering\includegraphics{asy_robot/fig2.pdf}%
\end{figure}
\pagebreak
Nel \textbf{terzo caso di esempio}, il percorso seguito è questo:
\begin{figure}[H]%
\centering\includegraphics{asy_robot/fig3.pdf}%
\end{figure}
