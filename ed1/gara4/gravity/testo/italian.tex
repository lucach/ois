% \usepackage{xcolor}
% \usepackage{afterpage}
\usepackage{pifont,mdframed}
% \usepackage[bottom]{footmisc}

\makeatletter
\gdef\this@inputfilename{input.txt}
\gdef\this@outputfilename{output.txt}
\makeatother

\newcommand{\inputfile}{\texttt{input.txt}}
\newcommand{\outputfile}{\texttt{output.txt}}

\newenvironment{warning}
  {\par\begin{mdframed}[linewidth=2pt,linecolor=gray]%
    \begin{list}{}{\leftmargin=1cm
                   \labelwidth=\leftmargin}\item[\Large\ding{43}]}
  {\end{list}\end{mdframed}\par}

	Gabriele è in visita alla stazione spaziale internazionale (\emph{ISS}), incaricato di aggiornare i sistemi di bordo ai nuovi standard del C$--$11. Purtroppo l'operazione è andata storta e la stazione è entrata in stato di freeze. Ora bisogna urgentemente riavviare tutti i sistemi di energy e life support prima che gli occupanti rischino la vita!

	Gabriele, imbracciata la tuta spaziale, esce sulla parete rettangolare esterna dell'astronave in cui si trovano i grossi interruttori di reset di emergenza di tutti i numerosi sistemi che compongono l'astronave. Gli $N$ pesanti interruttori si trovano sulla parete uno di seguito all'altro, ma ciascuno ad una altezza differente $H_i$. Ogni volta che spingerà un'interruttore per attivare il corrispondente sistema, Gabriele sa che la sua direzione di moto lungo l'asse verticale si invertirà (sbalzato dal principio di azione e reazione), mentre la sua direzione di moto lungo l'asse orizzontale rimarrà invariata. Il problema ora è quindi quello di pianificare una strategia per riavviare più sistemi possibili in una sola passata, guadagnandosi qualche minuto di tempo per poter stabilizzare la situazione!

	Aiuta Gabriele a spingere il maggior numero di interruttori. Una sottosequenza di interruttori è ammissibile se è \emph{alternante}, e cioè non vi sono tre interruttori di seguito (nella sequenza) con altezze tutte crescenti o tutte decrescenti. Trova quindi la più lunga sottosequenza ammissibile!

\Implementation
Dovrai sottoporre esattamente un file con estensione \texttt{.c}, \texttt{.cpp} o \texttt{.pas}.

\begin{warning}
Tra gli allegati a questo task troverai un template (\texttt{gravity.c}, \texttt{gravity.cpp}, \texttt{gravity.pas}) con un esempio di implementazione da completare.
\end{warning}

Se sceglierai di utilizzare il template, dovrai implementare la seguente funzione:
\begin{center}\begin{tabularx}{\textwidth}{|c|X|}
\hline
C/C++  & \verb|int passeggia(int N, int H[]);|\\
\hline
Pascal & \verb|function passeggia(N: longint; var H: array of longint): longint;|\\
\hline
\end{tabularx}\end{center}
In cui:
\begin{itemize}[nolistsep]
  \item L'intero $N$ rappresenta il numero di interruttori presenti sulla parete.
  \item L'array \texttt{H}, indicizzato da $0$ a $N-1$, contiene le altezze $H_i$ a cui gli interruttori sono situati.
  \item La funzione dovrà restituire la massima lunghezza per una sottosequenza alternante, che verrà stampata sul file di output.
\end{itemize}

\InputFile
Il file \inputfile{} è composto da due righe. La prima riga contiene l'unico intero $N$. La seconda riga contiene gli $N$ interi $H_i$ separati da uno spazio.

\OutputFile
Il file \outputfile{} è composto da un'unica riga contenente un unico intero, la risposta a questo problema.

% Assunzioni
\Constraints
\begin{itemize}[nolistsep, itemsep=2mm]
	\item $1 \le N \le 10\,000$.
	\item $1 \le H_i \le 20\,000$ per ogni $i=0\ldots N-1$.
	\item Le altezze $H_i$ sono tutte differenti.
\end{itemize}

\Scoring
Il tuo programma verrà testato su diversi test case raggruppati in subtask.
Per ottenere il punteggio relativo ad un subtask, è necessario risolvere
correttamente tutti i test relativi ad esso.

\begin{itemize}[nolistsep,itemsep=2mm]
  \item \textbf{\makebox[2cm][l]{Subtask 1} [10 punti]}: Casi d'esempio.
  \item \textbf{\makebox[2cm][l]{Subtask 2} [20 punti]}: $N \leq 10$.
  \item \textbf{\makebox[2cm][l]{Subtask 3} [40 punti]}: $N \leq 100$.
  \item \textbf{\makebox[2cm][l]{Subtask 4} [30 punti]}: Nessuna limitazione specifica.
\end{itemize}

% Esempi
\Examples
\begin{example}
\exmp{
4
1 2 3 4
}{%
2
}%
\end{example}
\begin{example}
\exmp{
8
5 7 6 4 1 8 3 2
}{%
5
}%
\end{example}


\Explanation
Nel \textbf{primo caso di esempio}, è possibile premere il primo e il secondo interruttore, ma non è possibile premerne tre.\\[2mm]
Nel \textbf{secondo caso di esempio}, è possibile premere gli interruttori 5-7-6-8-3.
