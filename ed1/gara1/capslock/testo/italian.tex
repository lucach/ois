% \usepackage{xcolor}
% \usepackage{afterpage}
\usepackage{pifont,mdframed}


\createsection{\Grader}{Grader di prova}

\newcommand{\inputfile}{\texttt{input.txt}}
\newcommand{\outputfile}{\texttt{output.txt}}

\newenvironment{warning}
  {\par\begin{mdframed}[linewidth=2pt,linecolor=gray]%
    \begin{list}{}{\leftmargin=1cm
                   \labelwidth=\leftmargin}\item[\Large\ding{43}]}
  {\end{list}\end{mdframed}\par}

Gabriele ha scritto un messaggio per Giorgio, ma arrivato alla fine si è accorto con orrore che tutto il testo ha le minuscole e le maiuscole invertite per colpa del caps lock attivo.

Piuttosto che riscrivere tutto daccapo, Gabriele decide di chiederti di creare un programma che, preso il testo del messaggio, converta le maiuscole in minuscole e viceversa.

Aiuta Gabriele ad aggiustare il messaggio!

\InputFile
Il file \inputfile{} è composto da due righe. La prima riga contiene l'unico intero $N$, il numero di caratteri del testo. La seconda riga contiene il testo del messaggio.

\OutputFile
Il file \outputfile{} è composto da un'unica riga contenente il testo corretto.

\Implementation
Dovrai sottoporre esattamente un file con estensione \texttt{.c}, \texttt{.cpp} o \texttt{.pas}.

\begin{warning}
Tra gli allegati a questo task troverai un template (\texttt{capslock.c}, \texttt{capslock.cpp}, \texttt{capslock.pas}) con un esempio di implementazione da completare.
\end{warning}

Se sceglierai di utilizzare il template, dovrai implementare la seguente funzione:
\begin{center}\begin{tabularx}{\textwidth}{|c|X|}
\hline
C/C++  & \verb|void aggiusta(int N, char S[]);|\\
\hline
Pascal & \verb|procedure aggiusta(N: longint; var S: array of char);|\\
\hline
\end{tabularx}\end{center}
In cui:
\begin{itemize}[nolistsep]
  \item L'intero $N$ rappresenta il numero di caratteri del testo.
  \item L'array \texttt{S}, indicizzato da $0$ a $N-1$, contiene i caratteri di cui il testo è composto. Questi possono essere:
  \begin{itemize}[nolistsep]
    \item una lettera (maiuscola o minuscola) dell'alfabeto;
  	\item uno spazio;
  	\item un segno di punteggiatura tra questi: \texttt{.,:;!?}.
  \end{itemize}
  \item La funzione dovrà trasformare la stringa $S$ nel testo corretto. Al termine dell'esecuzione della funzione verrà stampato sul file di output il nuovo contenuto dell'array $S$.
\end{itemize}

\Constraints 
\begin{itemize}[nolistsep,itemsep=2mm]
  \item $1 \le N \le 10\,000$.
\end{itemize}

\pagebreak
\Scoring
Il tuo programma verrà testato su diversi test case raggruppati in subtask.
Per ottenere il punteggio relativo ad un subtask, è necessario risolvere
correttamente tutti i test relativi ad esso.

\begin{itemize}[nolistsep,itemsep=2mm]
  \item \textbf{\makebox[2cm][l]{Subtask 1} [10 punti]}: Casi d'esempio.
  \item \textbf{\makebox[2cm][l]{Subtask 2} [20 punti]}: $1\le N \le 100$.
  \item \textbf{\makebox[2cm][l]{Subtask 3} [40 punti]}: $1 \le N \le 1000$.
  \item \textbf{\makebox[2cm][l]{Subtask 4} [30 punti]}: Nessuna limitazione specifica.
\end{itemize}

\Examples
\begin{example}
\exmp{
41
eHI, TUTTO BENE? tI VA UNA PIZZA STASERA?
}{%
Ehi, tutto bene? Ti va una pizza stasera?
}%
\end{example}
\begin{example}
\exmp{
23
nA nA nA nA nA bATmAN!!
}{%
Na Na Na Na Na BatMan!!
}%
\end{example}
