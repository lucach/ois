% \usepackage{xcolor}
% \usepackage{afterpage}
\usepackage{pifont,mdframed}
% \usepackage[bottom]{footmisc}

\makeatletter
\gdef\this@inputfilename{input.txt}
\gdef\this@outputfilename{output.txt}
\makeatother

\newcommand{\inputfile}{\texttt{input.txt}}
\newcommand{\outputfile}{\texttt{output.txt}}

\newenvironment{warning}
  {\par\begin{mdframed}[linewidth=2pt,linecolor=gray]%
    \begin{list}{}{\leftmargin=1cm
                   \labelwidth=\leftmargin}\item[\Large\ding{43}]}
  {\end{list}\end{mdframed}\par}

Gabriele e Giorgio hanno deciso di organizzare una festa. Per questo hanno raccolto una lista di tutti i loro amici, e ora vogliono invitarli. Affinché gli amici di Gabriele e Giorgio non si annoino alla festa, è necessario che ogni invitato conosca almeno altri due degli amici invitati.

Conoscendo il grafo delle conoscenze reciproche tra gli amici di Giorgio e Gabriele, stabilisci qual è il massimo numero di persone che Gabriele e Giorgio possono invitare alla festa.

\Implementation
Dovrai sottoporre esattamente un file con estensione \texttt{.c}, \texttt{.cpp} o \texttt{.pas}.

\begin{warning}
Tra gli allegati a questo task troverai un template (\texttt{festa.c}, \texttt{festa.cpp}, \texttt{festa.pas}) con un esempio di implementazione da completare.
\end{warning}

Se sceglierai di utilizzare il template, dovrai implementare la seguente funzione:
\begin{center}\begin{tabularx}{\textwidth}{|c|X|}
\hline
C/C++  & \verb|int invita(int N, int M, int conoscenzeA[], int conoscenzeB[]);|\\
\hline
Pascal & \footnotesize\verb|function invita(N, M: longint; var conoscenzeA, conoscenzeB: array of longint): longint;|\\
\hline
\end{tabularx}\end{center}
In cui:
\begin{itemize}[nolistsep]
  \item L'intero $N$ rappresenta il numero di amici di Gabriele e Giorgio.
  \item L'intero $M$ rappresenta il numero di archi nel grafo delle conoscenze (coppie di amici che si conoscono).
  \item Gli array \texttt{conoscenzeA} e \texttt{conoscenzeB}, indicizzati da $0$ a $M-1$, contengono le informazioni sugli archi del grafo: per ogni $0\le i < M$, gli amici \texttt{conoscenzeA}$[i]$ e \texttt{conoscenzeB}$[i]$ si conoscono. Gli amici sono indicizzati con valori $0, 1, \ldots, N-1$.
  \item La funzione dovrà restituire il massimo numero di persone che è possibile invitare rispettando la condizione che ogni invitato conosca almeno altri due invitati. Tale valore verrà stampato sul file di output.
\end{itemize}

\InputFile
Il file \inputfile{} è composto da $M + 1$ righe. La prima riga contiene gli interi $N$ e $M$ separati da uno spazio. Le righe successive contengono la descrizione degli archi del grafo delle conoscenze: sulla $i$-esima di queste righe sono presenti gli interi \texttt{conoscenzeA}$[i]$ e \texttt{conoscenzeB}$[i]$, separati da uno spazio.

\OutputFile
Il file \outputfile{} è composto da un'unica riga contenente un unico intero, la risposta a questo problema.

\pagebreak
% Assunzioni
\Constraints
\begin{itemize}[nolistsep, itemsep=2mm]
	\item $1 \le N \le 10\,000$.
	\item $0 \le M \le 100\,000$.	
	\item Il grafo delle conoscenze è un grafo non diretto (cioè la conoscenza è sempre reciproca). Tutti gli archi sono validi ($0 \le \texttt{conoscenzeA}[i], \texttt{conoscenzeB}[i] \le N-1$ e $\texttt{conoscenzeA}[i] \neq \texttt{conoscenzeB}[i]$), e non vengono ripetuti.
	\item Gabriele e Giorgio non vanno contati nel numero di amici conosciuti dagli invitati.
	\item Nel caso in cui non sia possibile invitare alcun amico, stampare il valore 0.
\end{itemize}

\Scoring
Il tuo programma verrà testato su diversi test case raggruppati in subtask.
Per ottenere il punteggio relativo ad un subtask, è necessario risolvere
correttamente tutti i test relativi ad esso.

\begin{itemize}[nolistsep,itemsep=2mm]
  \item \textbf{\makebox[2cm][l]{Subtask 1} [10 punti]}: Casi d'esempio.
  \item \textbf{\makebox[2cm][l]{Subtask 2} [20 punti]}: $N \leq 10$.
  \item \textbf{\makebox[2cm][l]{Subtask 3} [40 punti]}: $N \leq 1000$.
  \item \textbf{\makebox[2cm][l]{Subtask 4} [30 punti]}: Nessuna limitazione specifica.
\end{itemize}

% Esempi
\Examples
\begin{example}
\exmp{
6 5
0 2
1 2
3 4
3 5
5 4
}{%
3
}%
\end{example}
\begin{example}
\exmp{
3 2
0 2
1 2
}{%
0
}%
\end{example}
\begin{example}
\exmp{
9 10
0 1
2 0
5 2
4 5
3 4
2 3
7 2
6 5
0 5
1 8
}{%
5
}%
\end{example}


\Explanation
Nel \textbf{primo caso di esempio} il massimo numero di invitati si raggiunge invitando alla festa gli amici 3, 4 e 5.\\[2mm]
Nel \textbf{secondo caso di esempio} non è possibile invitare alcun amico alla festa.\\[2mm]
Nel \textbf{terzo caso di esempio} il massimo numero di invitati si raggiunge invitando alla festa gli amici 3, 4, 2, 0, 5.
