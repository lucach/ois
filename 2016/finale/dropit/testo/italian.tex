\usepackage{xcolor}
\usepackage{afterpage}
\usepackage{pifont,mdframed}
\usepackage[bottom]{footmisc}

\makeatletter
\gdef\this@inputfilename{input.txt}
\gdef\this@outputfilename{output.txt}
\makeatother

\newcommand{\inputfile}{\texttt{input.txt}}
\newcommand{\outputfile}{\texttt{output.txt}}

\newenvironment{warning}
  {\par\begin{mdframed}[linewidth=2pt,linecolor=gray]%
    \begin{list}{}{\leftmargin=1cm
                   \labelwidth=\leftmargin}\item[\Large\ding{43}]}
  {\end{list}\end{mdframed}\par}

	William ha scaricato il gioco \emph{Drop-It!} per il suo cellulare. In questo gioco alcuni blocchi di varia lunghezza vengono fatti cadere da una gru uno sull'altro, e cadendo si comportano nel seguente modo:
	\begin{enumerate}
		\item Se un blocco cade su uno lungo uguale, si ferma distruggendosi assieme a quello su cui è caduto;
		\item Se un blocco cade su uno pi\`u lungo, si ferma l\`i e sopra di esso se ne crea uno lungo quanto la differenza tra i due che si sono scontrati;
		\item Se un blocco cade su uno pi\`u piccolo, si riduce della lunghezza di quello pi\`u piccolo, distrugge il pi\`u piccolo e continua a cadere su quello successivo.
	\end{enumerate}
	Per aiutare William a pianificare meglio la strategia, scrivi un programma che calcoli la situazione della pila dopo la caduta di $N$ blocchi con lunghezze $L_i$ per $i=0, \ldots, N$.

\Implementation
Dovrai sottoporre esattamente un file con estensione \texttt{.c}, \texttt{.cpp} o \texttt{.pas}.

\begin{warning}
Tra gli allegati a questo task troverai un template (\texttt{dropit.c}, \texttt{dropit.cpp}, \texttt{dropit.pas}) con un esempio di implementazione da completare.
\end{warning}

Se sceglierai di utilizzare il template, dovrai implementare la seguente funzione:
\begin{center}\begin{tabularx}{\textwidth}{|c|X|}
\hline
C/C++  & \verb|int cadi(int N, int L[], int P[]);|\\
\hline
Pascal & \verb|function cadi(N: longint; var L, P: array of longint): longint;|\\
\hline
\end{tabularx}\end{center}
In cui:
\begin{itemize}[nolistsep]
  \item L'intero $N$ rappresenta il numero di blocchi che vengono fatti cadere dalla gru.
  \item L'array \texttt{L}, indicizzato da $0$ a $N-1$, contiene le lunghezze dei pezzi fatti cadere dalla gru nell'ordine in cui vengono fatti cadere.
  \item L'array \texttt{P}, indicizzato da $0$ a $M-1$, dovr\`a essere riempito dalla funzione con le lunghezze dei blocchi che rimangono impilati alla fine del procedimento.
  \item La funzione dovrà restituire il numero $M$ di blocchi rimasti alla fine, che verrà stampato sul file di output assieme all'array \texttt{P}.
\end{itemize}

\InputFile
Il file \inputfile{} è composto da due righe. La prima riga contiene l'unico intero $N$. La seconda riga contiene gli $N$ interi $L_i$ separati da uno spazio.

\OutputFile
Il file \outputfile{} è composto da due righe. La prima riga contiene l'unico intero $M$. La seconda riga contiene gli $N$ interi $P_i$ separati da uno spazio.

% Assunzioni
\Constraints
\begin{itemize}[nolistsep, itemsep=2mm]
	\item $1 \le N \le 100\,000$.
	\item $1 \le L_i \le 10\,000$ per ogni $i=0\ldots N-1$.
\end{itemize}

\Scoring
Il tuo programma verrà testato su diversi test case raggruppati in subtask.
Per ottenere il punteggio relativo ad un subtask, è necessario risolvere
correttamente tutti i test relativi ad esso.

\begin{itemize}[nolistsep,itemsep=2mm]
  \item \textbf{\makebox[2cm][l]{Subtask 1} [10 punti]}: Casi d'esempio.
  \item \textbf{\makebox[2cm][l]{Subtask 2} [20 punti]}: $N \leq 10$.
  \item \textbf{\makebox[2cm][l]{Subtask 3} [40 punti]}: $N \leq 1000$.
  \item \textbf{\makebox[2cm][l]{Subtask 4} [30 punti]}: Nessuna limitazione specifica.
\end{itemize}

% Esempi


\Examples
\begin{example}
\exmpfile{dropit.input0.txt}{dropit.output0.txt}%
\exmpfile{dropit.input1.txt}{dropit.output1.txt}%
\end{example}


\Explanation
Nel \textbf{primo caso di esempio}, si applica prima la regola (1) svuotando la pila finora formata e infine la regola (2) creando un ulteriore blocco lungo $22$.\\[2mm]
Nel \textbf{secondo caso di esempio}, si applicano le regole (2), (1), (3) e (2) ottenendo la successione di stati:
\begin{center}
\texttt{17 <---- 13\\
17 13 4 <---- 4\\
17 13  <---- 15\\
17 <---- 2\\
17 2 15
}
\end{center}
