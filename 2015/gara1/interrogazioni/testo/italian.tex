% \usepackage{xcolor}
% \usepackage{afterpage}
\usepackage{pifont,mdframed}
% \usepackage[bottom]{footmisc}

\makeatletter
\gdef\this@inputfilename{input.txt}
\gdef\this@outputfilename{output.txt}
\makeatother

\createsection{\Grader}{Grader di prova}

\newcommand{\inputfile}{\texttt{input.txt}}
\newcommand{\outputfile}{\texttt{output.txt}}

\newenvironment{warning}
  {\par\begin{mdframed}[linewidth=2pt,linecolor=gray]%
    \begin{list}{}{\leftmargin=1cm
                   \labelwidth=\leftmargin}\item[\Large\ding{43}]}
  {\end{list}\end{mdframed}\par}

Giulia, la professoressa di storia dell'arte di Gabriele, ha deciso di interrogare $K$ persone oggi. Rivolgerà ad ogni interrogato una domanda, presa da una lista (segreta!) di $N$ quesiti di varia difficoltà. La professoressa sa bene che le domande rivolte agli studenti dovranno essere tutte di difficoltà comparabile, per evitare il malcontento della classe. La scontentezza della classe infatti è pari alla differenza tra la difficoltà della domanda più difficile e quella più facile tra quelle chieste. 

Se Giulia sceglie $K$ domande dalla lista nel modo ottimo, qual è la minima scontentezza possibile della classe?

\InputFile
Il file \inputfile{} è composto da due righe. La prima riga contiene i due interi $N$ e $K$ separati da uno spazio. La seconda riga contiene $N$ interi separati da uno spazio, le difficoltà $D_i$ delle domande della lista.

\OutputFile
Il file \outputfile{} è composto da un'unica riga contenente un unico intero, la risposta a questo problema.


\Implementation
Dovrai sottoporre esattamente un file con estensione \texttt{.c}, \texttt{.cpp} o \texttt{.pas}.

\begin{warning}
Tra gli allegati a questo task troverai un template (\texttt{interrogazioni.c}, \texttt{interrogazioni.cpp}, \texttt{interrogazioni.pas}) con un esempio di implementazione.
\end{warning}

Dovrai implementare la seguente funzione:
\begin{center}\begin{tabularx}{\textwidth}{|c|X|}
\hline
C/C++  & \verb|int interroga(int N, int K, int D[]);|\\
\hline
Pascal & \verb|function interroga(N, K: longint; var D: array of longint): longint;|\\
\hline
\end{tabularx}\end{center}
In cui:
\begin{itemize}[nolistsep]
  \item L'intero $N$ rappresenta il numero totale di domande nella lista.
  \item L'intero $K$ rappresenta il numero di domande da selezionare.
  \item L'array \texttt{D}, indicizzato da $0$ a $N-1$, contiene le difficoltà delle domande della lista.
  \item La funzione dovrà restituire la minima scontentezza possibile della classe, che verrà stampata sul file di output.
\end{itemize}

\Constraints 
\begin{itemize}[nolistsep,itemsep=2mm]
  \item $1 \le K \le N \le 10\,000$.
  \item $1 \le D_i \le 100\,000$ per ogni $i=0\ldots N-1$.
\end{itemize}

\Scoring
Il tuo programma verrà testato su diversi test case raggruppati in subtask.
Per ottenere il punteggio relativo ad un subtask, è necessario risolvere
correttamente tutti i test relativi ad esso.

\begin{itemize}[nolistsep,itemsep=2mm]
  \item \textbf{\makebox[2cm][l]{Subtask 1} [10 punti]}: Casi d'esempio.
  \item \textbf{\makebox[2cm][l]{Subtask 2} [20 punti]}: $1\le N \le 10$.
  \item \textbf{\makebox[2cm][l]{Subtask 3} [10 punti]}: $1 \le N \le 1000$, $K = 2$.
  \item \textbf{\makebox[2cm][l]{Subtask 4} [30 punti]}: $1 \le N \le 1000$.
  \item \textbf{\makebox[2cm][l]{Subtask 5} [30 punti]}: Nessuna limitazione specifica.
\end{itemize}

\Examples
\begin{example}
\exmp{
5 2
4 10 7 12 9 
}{%
1
}%
\end{example}
\begin{example}
\exmp{
6 3
12 34 54 59 29 44
}{%
15
}%
\end{example}


\Explanation
Nel \textbf{primo caso di esempio} conviene selezionare le domande di difficoltà 9 e 10.\\[2mm]
Nel \textbf{secondo caso di esempio} conviene selezionare le domande di difficoltà 34, 29 e 44 oppure le domande di difficoltà 54, 59 e 44.
