% \usepackage{xcolor}
% \usepackage{afterpage}
\usepackage{pifont,mdframed}
% \usepackage[bottom]{footmisc}

\makeatletter
\gdef\this@inputfilename{input.txt}
\gdef\this@outputfilename{output.txt}
\makeatother

\createsection{\Grader}{Grader di prova}

\newcommand{\inputfile}{\texttt{input.txt}}
\newcommand{\outputfile}{\texttt{output.txt}}

\newenvironment{warning}
  {\par\begin{mdframed}[linewidth=2pt,linecolor=gray]%
    \begin{list}{}{\leftmargin=1cm
                   \labelwidth=\leftmargin}\item[\Large\ding{43}]}
  {\end{list}\end{mdframed}\par}
  
Il professore di disegno tecnico di Gabriele è un uomo stravagante. I ragazzi hanno scoperto che il professore ama moltissimo i quadrati, e che spesso per prendere un buon voto in un compito è sufficiente che la tavola ne contenga uno. Tuttavia non tutti i quadrati sono uguali agli occhi del professore: sono per lui gradevoli alla vista solo quelli con area compresa tra i valori $A$ e $B$ (inclusi).

Quanti sono i quadrati diversi che il professore trova gradevoli?

\InputFile
Il file \inputfile{} ha questo formato:
\begin{itemize}[nolistsep,itemsep=2mm]
\item Riga $1$: contiene gli interi $A$ e $B$.
\end{itemize}

\OutputFile
Il file \outputfile{} ha questo formato:
\begin{itemize}[nolistsep,itemsep=2mm]
\item Riga $1$: contiene la risposta al problema.
\end{itemize}

\Implementation
Dovrai sottoporre esattamente un file con estensione \texttt{.c}, \texttt{.cpp} o \texttt{.pas}.

\begin{warning}
Tra gli allegati a questo task troverai un template (\texttt{quadrati.c}, \texttt{quadrati.cpp}, \texttt{quadrati.pas}) con un esempio di implementazione da completare.
\end{warning}

Se sceglierai di utilizzare il template, dovrai implementare la seguente funzione:
\begin{center}\begin{tabularx}{\textwidth}{|c|X|}
\hline
C/C++  & \verb|int conta(int A, int B);|\\
\hline
Pascal & \verb|function conta(A, B: longint): longint;|\\
\hline
\end{tabularx}\end{center}
La funzione riceverà come parametri i valori $A$ e $B$ e dovrà ritornare la risposta al problema, che verrà stampata sul file di output.

% Assunzioni
\Constraints
\begin{itemize}[nolistsep, itemsep=2mm]
\item $1 \le A \le B \le 1\,000\,000 $.
\end{itemize}

\pagebreak
\Scoring
Il tuo programma verrà testato su diversi test case raggruppati in subtask.
Per ottenere il punteggio relativo ad un subtask, è necessario risolvere
correttamente tutti i test relativi ad esso.

\begin{itemize}[nolistsep,itemsep=2mm]
  \item \textbf{\makebox[2cm][l]{Subtask 1} [10 punti]}: Casi d'esempio.
  \item \textbf{\makebox[2cm][l]{Subtask 2} [20 punti]}: $A \le B \le 100$.
  \item \textbf{\makebox[2cm][l]{Subtask 3} [40 punti]}: $A \le B \le 1\,000$.
  \item \textbf{\makebox[2cm][l]{Subtask 4} [30 punti]}: Nessuna limitazione specifica.
\end{itemize}

% Esempi
\Examples
\begin{example}
\exmp{
6 16
}{ %
2
} %
\end{example}
\begin{example}
\exmp{
4 5
}{ %
1
} %
\end{example}