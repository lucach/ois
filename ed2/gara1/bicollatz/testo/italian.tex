\usepackage{xcolor}
\usepackage{afterpage}
\usepackage{pifont,mdframed}
\usepackage[bottom]{footmisc}

\makeatletter
\gdef\this@inputfilename{input.txt}
\gdef\this@outputfilename{output.txt}
\makeatother

\newcommand{\inputfile}{\texttt{input.txt}}
\newcommand{\outputfile}{\texttt{output.txt}}

\newenvironment{warning}
  {\par\begin{mdframed}[linewidth=2pt,linecolor=gray]%
    \begin{list}{}{\leftmargin=1cm
                   \labelwidth=\leftmargin}\item[\Large\ding{43}]}
  {\end{list}\end{mdframed}\par}

	Dopo lunghi e accurati studi sulla congettura di \emph{Lollatz}, Giorgio ha deciso che è giunto finalmente il momento di creare una nuova congettura egli stesso. Propone quindi la nuova congettura di \emph{Bicollatz}, descritta dal seguente procedimento. Dati due numeri interi positivi $A$ e $B$, si ripete il seguente passaggio:
	\begin{itemize}
		\item se $A$ e $B$ sono entrambi pari, si dividono entrambi per due;
		\item se $A$ e $B$ sono entrambi dispari, ciascuno dei due si moltiplica per tre e si aggiunge uno;
		\item altrimenti, si aggiunge $3$ a quello dei due che è dispari.
	\end{itemize}
	La congettura dice che ripetendo questo passaggio, si arriva sempre prima o poi a ottenere che $A,B = (1,1)$. Aiuta Giorgio a verificare la sua congettura!

\Implementation
Dovrai sottoporre esattamente un file con estensione \texttt{.c}, \texttt{.cpp} o \texttt{.pas}.

\begin{warning}
Tra gli allegati a questo task troverai un template (\texttt{bicollatz.c}, \texttt{bicollatz.cpp}, \texttt{bicollatz.pas}) con un esempio di implementazione da completare.
\end{warning}

Se sceglierai di utilizzare il template, dovrai implementare la seguente funzione:
\begin{center}\begin{tabularx}{\textwidth}{|c|X|}
\hline
C/C++  & \verb|int bicollatz(int A, int B);|\\
\hline
Pascal & \verb|function bicollatz(A, B: longint): longint;|\\
\hline
\end{tabularx}\end{center}
In cui:
\begin{itemize}[nolistsep]
  \item Gli interi $A$ e $B$ rappresentano gli interi positivi iniziali.
  \item La funzione dovrà restituire il numero di passaggi prima di ottenere $(1,1)$, oppure $-1$ se la congettura è falsa per $A$,$B$, e questo valore verrà stampato sul file di output.
\end{itemize}

\InputFile
Il file \inputfile{} è composto da un'unica riga contenente i due interi $A$ e $B$.

\OutputFile
Il file \outputfile{} è composto da un'unica riga contenente un unico intero, la risposta a questo problema.

% Assunzioni
\Constraints
\begin{itemize}[nolistsep, itemsep=2mm]
	\item $1 \le A, B \le 1\,000\,000$.
\end{itemize}

\Scoring
Il tuo programma verrà testato su diversi test case raggruppati in subtask.
Per ottenere il punteggio relativo ad un subtask, è necessario risolvere
correttamente tutti i test relativi ad esso.

\begin{itemize}[nolistsep,itemsep=2mm]
  \item \textbf{\makebox[2cm][l]{Subtask 1} [10 punti]}: Casi d'esempio.
  \item \textbf{\makebox[2cm][l]{Subtask 2} [20 punti]}: $N \leq 100$.
  \item \textbf{\makebox[2cm][l]{Subtask 3} [40 punti]}: $N \leq 10\,000$.
  \item \textbf{\makebox[2cm][l]{Subtask 4} [30 punti]}: Nessuna limitazione specifica.
\end{itemize}

% Esempi
\Examples
\begin{example}
\exmpfile{bicollatz.input0.txt}{bicollatz.output0.txt}%
\exmpfile{bicollatz.input1.txt}{bicollatz.output1.txt}%
\end{example}


\Explanation
Nel \textbf{primo caso di esempio}, $A,B = (1,1)$ senza dover effettuare alcun passaggio.\\[2mm]
Nel \textbf{secondo caso di esempio}, si ottiene la seguente sequenza: $(14,1)$, $(14,4)$, $(7,2)$, $(10,2)$, $(5,1)$, $(16,4)$, $(8,2)$, $(4,1)$, $(4,4)$, $(2,2)$, $(1,1)$.
