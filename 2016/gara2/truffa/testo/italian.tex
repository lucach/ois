\usepackage{xcolor}
\usepackage{afterpage}
\usepackage{pifont,mdframed}
\usepackage[bottom]{footmisc}

\makeatletter
\gdef\this@inputfilename{input.txt}
\gdef\this@outputfilename{output.txt}
\makeatother

\newcommand{\inputfile}{\texttt{input.txt}}
\newcommand{\outputfile}{\texttt{output.txt}}

\newenvironment{warning}
  {\par\begin{mdframed}[linewidth=2pt,linecolor=gray]%
    \begin{list}{}{\leftmargin=1cm
                   \labelwidth=\leftmargin}\item[\Large\ding{43}]}
  {\end{list}\end{mdframed}\par}

	La \emph{SteamPower S.P.A.}, azienda leader mondiale nel campo delle macchine a vapore portatili, è ancora una volta in crisi nonostante le oculate manovre messe in atto nell'anno passato. Ora è il momento di stilare il bilancio di fine anno, che è di nuovo in passivo. Per non turbare gli azionisti, il \emph{CEO} ha ricontattato il massimo esperto mondiale in campo di finanza creativa (il cui nome non possiamo rivelare).

	Questa volta l'esperto ha elaborato un nuovo stratagemma: con un'audace manovra detta \emph{``la sfangata''} una voce in uscita può diventare una voce in entrata. Questa manovra può essere ripetuta fino a che il bilancio non diventi in attivo, ma per minimizzare i rischi conviene effettuarla il \emph{minor numero di volte}. Quante volte al minimo è necessario effettuare una sfangata affinché il bilancio diventi in attivo?

\Implementation
Dovrai sottoporre esattamente un file con estensione \texttt{.c}, \texttt{.cpp} o \texttt{.pas}.

\begin{warning}
Tra gli allegati a questo task troverai un template (\texttt{truffa.c}, \texttt{truffa.cpp}, \texttt{truffa.pas}) con un esempio di implementazione da completare.
\end{warning}

Se sceglierai di utilizzare il template, dovrai implementare la seguente funzione:
\begin{center}\begin{tabularx}{\textwidth}{|c|X|}
\hline
C/C++  & \verb|int sfangate(int N, int V[]);|\\
\hline
Pascal & \verb|function sfangate(N: longint; var V: array of longint): longint;|\\
\hline
\end{tabularx}\end{center}
In cui:
\begin{itemize}[nolistsep]
  \item L'intero $N$ rappresenta il numero di voci del bilancio.
  \item L'array \texttt{V}, indicizzato da $0$ a $N-1$, contiene le voci $V_i$ del bilancio (positive le entrate e negative le uscite).
  \item La funzione dovrà restituire il minor numero possibile di sfangate necessarie a rendere il bilancio in attivo, che verrà stampato sul file di output.
\end{itemize}

\InputFile
Il file \inputfile{} è composto da due righe. La prima riga contiene l'unico intero $N$. La seconda riga contiene gli $N$ interi $V_i$ separati da uno spazio.

\OutputFile
Il file \outputfile{} è composto da un'unica riga contenente un unico intero, la risposta a questo problema.

% Assunzioni
\Constraints
\begin{itemize}[nolistsep, itemsep=2mm]
	\item $1 \le N \le 100\,000$.
	\item $-10\,000 \le V_i \le 10\,000$, $V_i \ne 0$ per ogni $i=0\ldots N-1$.
\end{itemize}

\Scoring
Il tuo programma verrà testato su diversi test case raggruppati in subtask.
Per ottenere il punteggio relativo ad un subtask, è necessario risolvere
correttamente tutti i test relativi ad esso.

\begin{itemize}[nolistsep,itemsep=2mm]
  \item \textbf{\makebox[2cm][l]{Subtask 1} [10 punti]}: Casi d'esempio.
  \item \textbf{\makebox[2cm][l]{Subtask 2} [20 punti]}: $N \leq 10$.
  \item \textbf{\makebox[2cm][l]{Subtask 3} [40 punti]}: $N \leq 1000$.
  \item \textbf{\makebox[2cm][l]{Subtask 4} [30 punti]}: Nessuna limitazione specifica.
\end{itemize}

% Esempi


\Examples
\begin{example}
\exmpfile{truffa.input0.txt}{truffa.output0.txt}%
\exmpfile{truffa.input1.txt}{truffa.output1.txt}%
\exmpfile{truffa.input2.txt}{truffa.output2.txt}%
\end{example}


\Explanation
Nel \textbf{primo caso di esempio}, è sufficiente ``sfangare'' la voce $-10\,000$ in $10\,000$.\\[2mm]
Nel \textbf{secondo caso di esempio}, si possono ad esempio ``sfangare'' le voci $-5000$ e $-1800$.\\[2mm]
Nel \textbf{terzo caso di esempio}, si deve ``sfangare'' la voce $-700$.
