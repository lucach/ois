\usepackage{xcolor}
\usepackage{afterpage}
\usepackage{pifont,mdframed}
\usepackage[bottom]{footmisc}

\makeatletter
\gdef\this@inputfilename{input.txt}
\gdef\this@outputfilename{output.txt}
\makeatother

\newcommand{\inputfile}{\texttt{input.txt}}
\newcommand{\outputfile}{\texttt{output.txt}}

\newenvironment{warning}
  {\par\begin{mdframed}[linewidth=2pt,linecolor=gray]%
    \begin{list}{}{\leftmargin=1cm
                   \labelwidth=\leftmargin}\item[\Large\ding{43}]}
  {\end{list}\end{mdframed}\par}

	William si è di recente appassionato a \emph{Impila la pila}, un nuovo gioco per il suo telefonino. In questo gioco ci sono $N$ interruttori in sequenza, numerati da $0$ a $N-1$ da sinistra verso destra, e un personaggio che all'inizio è situato nell'angolo sinistro dello schermo (in corrispondenza dell'interruttore $0$). Durante il gioco è possibile ordinare al personaggio di muoversi verso destra o verso sinistra (e queste operazioni richiedono un secondo) oppure di premere l'interruttore davanti al quale è situato (e questa operazione è istantanea).

	A rendere il gioco interessante, sopra a ogni interruttore è posizionato un disco con una certa dimensione di base $B_i$. Ogni volta che viene premuto un interruttore, il disco corrispondente viene aggiunto a una pila che cresce quindi progressivamente. Lo scopo del gioco è di attivare tutti gli interruttori nell'ordine corretto, di modo che la pila che si viene a formare sia costituita da dischi con basi sempre strettamente decrescenti.

	Willam ha appena finito un livello e non sa se affrontare anche quello successivo: tra pochi minuti deve presentarsi a una lezione, e non vuole assolutamente essere costretto a lasciarlo a metà. Aiutalo a calcolare di quanti secondi ha bisogno per completarlo, così che sia in grado di decidere se fa in tempo oppure no!

\Implementation
Dovrai sottoporre esattamente un file con estensione \texttt{.c}, \texttt{.cpp} o \texttt{.pas}.

\begin{warning}
Tra gli allegati a questo task troverai un template (\texttt{combinazione.c}, \texttt{combinazione.cpp}, \texttt{combinazione.pas}) con un esempio di implementazione da completare.
\end{warning}

Se sceglierai di utilizzare il template, dovrai implementare la seguente funzione:
\begin{center}\begin{tabularx}{\textwidth}{|c|X|}
\hline
C/C++  & \verb|long long premi(int N, int B[]);|\\
\hline
Pascal & \verb|function premi(N: longint; var B: array of longint): int64;|\\
\hline
\end{tabularx}\end{center}
In cui:
\begin{itemize}[nolistsep]
  \item L'intero $N$ rappresenta il numero di interruttori.
  \item L'array \texttt{B}, indicizzato da $0$ a $N-1$, contiene per ogni $i$ la dimensione di base $B_i$ del disco associato all'interruttore $i$-esimo.
  \item La funzione dovrà restituire il numero minimo di secondi necessario a completare il livello, che verrà stampato sul file di output.
\end{itemize}

\InputFile
Il file \inputfile{} è composto da due righe. La prima riga contiene l'unico intero $N$. La seconda riga contiene gli $N$ interi $B_i$ separati da uno spazio.

\OutputFile
Il file \outputfile{} è composto da un'unica riga contenente un unico intero, la risposta a questo problema.

% Assunzioni
\Constraints
\begin{itemize}[nolistsep, itemsep=2mm]
	\item $1 \le N \le 100\,000$.
	\item $1 \le B_i \le 100\,000$ per ogni $i=0\ldots N-1$.
	\item I valori $B_i$ sono tutti differenti, ossia non esistono duplicati.
\end{itemize}

\Scoring
Il tuo programma verrà testato su diversi test case raggruppati in subtask.
Per ottenere il punteggio relativo ad un subtask, è necessario risolvere
correttamente tutti i test relativi ad esso.

\begin{itemize}[nolistsep,itemsep=2mm]
  \item \textbf{\makebox[2cm][l]{Subtask 1} [10 punti]}: Casi d'esempio.
  \item \textbf{\makebox[2cm][l]{Subtask 2} [20 punti]}: $N \leq 10$.
  \item \textbf{\makebox[2cm][l]{Subtask 3} [40 punti]}: $N \leq 1000$.
  \item \textbf{\makebox[2cm][l]{Subtask 4} [30 punti]}: Nessuna limitazione specifica.
\end{itemize}

% Esempi


\Examples
\begin{example}
\exmpfile{combinazione.input0.txt}{combinazione.output0.txt}%
\exmpfile{combinazione.input1.txt}{combinazione.output1.txt}%
\end{example}


\Explanation
Nel \textbf{primo caso di esempio}, in un secondo si raggiunge l'interruttore con base $29$, in un altro secondo l'interruttore con base $21$ e in altri due secondi l'interruttore con base $13$.\\[2mm]
Nel \textbf{secondo caso di esempio}, in 6 secondi si raggiunge l'interruttore con base $17$, e in altri 6 secondi si scandiscono tutti gli altri interruttori da destra verso sinistra.
