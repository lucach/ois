% \usepackage{xcolor}
% \usepackage{afterpage}
\usepackage{pifont,mdframed}
% \usepackage[bottom]{footmisc}

\makeatletter
\gdef\this@inputfilename{input.txt}
\gdef\this@outputfilename{output.txt}
\makeatother

\newcommand{\inputfile}{\texttt{input.txt}}
\newcommand{\outputfile}{\texttt{output.txt}}

\newenvironment{warning}
  {\par\begin{mdframed}[linewidth=2pt,linecolor=gray]%
    \begin{list}{}{\leftmargin=1cm
                   \labelwidth=\leftmargin}\item[\Large\ding{43}]}
  {\end{list}\end{mdframed}\par}

	Gabriele deve andare a trovare Giorgio per mettere a punto la finale delle \emph{OIS}. La strada da Brescia a Pinerolo è lunga, e attraversa $N+1$ città (numerate da 0 a $N$) tutte alla stessa distanza di $1$ tantometro. Dato che Gabriele ha recentemente racimolato un po' di liquidità, sta valutando se viaggiare comodo spostandosi di città in città in taxi. In ogni città c'è una stazione di taxi, ma la situazione è resa complicata dal fatto che i prezzi dei taxi variano notevolmente di città in città. Prendere un taxi nella città $i$ ha un prezzo base di $C_i$ euro per il primo tantometro percorso, e poi questo prezzo aumenta di uno per ogni ulteriore tantometro percorso (i tassisti sono reticenti ad allontanarsi dalla loro città natìa!).

	Per esempio, nella seguente situazione:

	\begin{center}
		\includegraphics[width = .9\textwidth]{asy_taxi/esempio.pdf}
	\end{center}

	Gabriele inizia prendendo il taxi a Brescia, la città numero 0, con un prezzo base di 10 euro al tantometro. Quindi lo utilizza per andare fino alla città numero 3, pagando $10+11+12 = 33$ euro. A questo punto, preferisce cambiare taxi e per arrivare a Pinerolo spende ancora $11$ euro, per un totale di $44$.

	Aiuta Gabriele a valutare se può permettersi di viaggiare in taxi, calcolando quanto costerebbe al minimo un tragitto da Brescia (città numero 0) a Pinerolo (città numero $N$) in taxi!

\Implementation
Dovrai sottoporre esattamente un file con estensione \texttt{.c}, \texttt{.cpp} o \texttt{.pas}.

\begin{warning}
Tra gli allegati a questo task troverai un template (\texttt{taxi.c}, \texttt{taxi.cpp}, \texttt{taxi.pas}) con un esempio di implementazione da completare.
\end{warning}

Se sceglierai di utilizzare il template, dovrai implementare la seguente funzione:
\begin{center}\begin{tabularx}{\textwidth}{|c|X|}
\hline
C/C++  & \verb|int viaggia(int N, int C[]);|\\
\hline
Pascal & \verb|function viaggia(N: longint; var C: array of longint): longint;|\\
\hline
\end{tabularx}\end{center}
In cui:
\begin{itemize}[nolistsep]
  \item L'intero $N$ rappresenta il numero di città tra Brescia e Pinerolo (incluse).
  \item L'array \texttt{C}, indicizzato da $0$ a $N-1$, contiene i prezzi base dei taxi nelle varie città. Non è riportato il prezzo dei taxi a Pinerolo, che non è rilevante per la soluzione di questo problema.
  \item La funzione dovrà restituire il costo minimo in euro per un tragitto in taxi da $0$ a $N$, che verrà stampato sul file di output.
\end{itemize}

\InputFile
Il file \inputfile{} è composto da due righe. La prima riga contiene l'unico intero $N$. La seconda riga contiene gli $N$ interi $C_i$ separati da uno spazio.

\OutputFile
Il file \outputfile{} è composto da un'unica riga contenente un unico intero, la risposta a questo problema.

% Assunzioni
\Constraints
\begin{itemize}[nolistsep, itemsep=2mm]
	\item $1 \le N \le 10\,000$.
	\item $1 \le C_i \le 100\,000$ per ogni $i=0\ldots N-1$.
\end{itemize}

\Scoring
Il tuo programma verrà testato su diversi test case raggruppati in subtask.
Per ottenere il punteggio relativo ad un subtask, è necessario risolvere
correttamente tutti i test relativi ad esso.

\begin{itemize}[nolistsep,itemsep=2mm]
  \item \textbf{\makebox[2cm][l]{Subtask 1} [10 punti]}: Casi d'esempio.
  \item \textbf{\makebox[2cm][l]{Subtask 2} [20 punti]}: $N \leq 10$.
  \item \textbf{\makebox[2cm][l]{Subtask 3} [40 punti]}: $N \leq 100$.
  \item \textbf{\makebox[2cm][l]{Subtask 4} [30 punti]}: Nessuna limitazione specifica.
\end{itemize}

% Esempi
\Examples
\begin{example}
\exmp{
4
10 15 12 11
}{%
44
}%
\end{example}
\begin{example}
\exmp{
12
27 21 99 35 71 23 64 5 10 44 1 1
}{%
184
}%
\end{example}


\Explanation
Nel \textbf{primo caso di esempio}, Gabriele prende i taxi nelle città 0 e 3.\\[2mm]
Nel \textbf{secondo caso di esempio}, Gabriele prende i taxi nelle città 0, 1, 5, 7, 10, 11.
