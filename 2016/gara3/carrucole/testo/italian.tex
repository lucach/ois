\usepackage{xcolor}
\usepackage{afterpage}
\usepackage{pifont,mdframed}
\usepackage[bottom]{footmisc}

\makeatletter
\gdef\this@inputfilename{input.txt}
\gdef\this@outputfilename{output.txt}
\makeatother

\newcommand{\inputfile}{\texttt{input.txt}}
\newcommand{\outputfile}{\texttt{output.txt}}

\newenvironment{warning}
  {\par\begin{mdframed}[linewidth=2pt,linecolor=gray]%
    \begin{list}{}{\leftmargin=1cm
                   \labelwidth=\leftmargin}\item[\Large\ding{43}]}
  {\end{list}\end{mdframed}\par}

	Giorgio e William vorrebbero andare a visitare il pi\`u grande parco avventura del mondo, composto da ben $N$ alberi collegati tra loro da $M$ tiri di carrucole (ogni tiro di carrucola collega una coppia di alberi). Purtroppo, un violento Tsunami si \`e abbattuto sul parco, abbattendo alcuni degli alberi. Naturalmente, per ogni albero abbattuto si sono persi tutti i tiri di carrucola che partivano da quell'albero. Giorgio e William si chiedono quindi se valga ancora la pena di pagare l'esoso biglietto di ingresso del parco, visto lo stato in cui si trova al momento. Aiutali calcolando quanti tiri di carrucole sono ancora rimasti in piedi!

\Implementation
Dovrai sottoporre esattamente un file con estensione \texttt{.c}, \texttt{.cpp} o \texttt{.pas}.

\begin{warning}
Tra gli allegati a questo task troverai un template (\texttt{carrucole.c}, \texttt{carrucole.cpp}, \texttt{carrucole.pas}) con un esempio di implementazione da completare.
\end{warning}

Se sceglierai di utilizzare il template, dovrai implementare la seguente funzione:
\begin{center}\begin{tabularx}{\textwidth}{|c|X|}
\hline
C/C++  & \verb|int conta(int N, int M, int B[], int da[], int a[]);|\\
\hline
Pascal & \verb|function conta(N, M: longint; var B, da, a: array of longint): longint;|\\
\hline
\end{tabularx}\end{center}
In cui:
\begin{itemize}[nolistsep]
  \item L'intero $N$ rappresenta il numero (iniziale) di alberi.
  \item L'intero $M$ rappresenta il numero (iniziale) di tiri di carrucole.
  \item L'array \texttt{B}, indicizzato da $0$ a $N-1$, contiene per ogni albero $0$ se l'albero \`e stato abbattuto oppure $1$ se l'albero \`e ancora in piedi.
  \item Gli array \texttt{da} e \texttt{a}, indicizzati da $0$ a $M-1$, descrivono i tiri di carrucola (che per ogni $i$ collegano \texttt{da[$i$]} e \texttt{a[$i$]}).
  \item La funzione dovrà restituire il numero di tiri di carrucola ancora rimasti in piedi, che verrà stampato sul file di output.
\end{itemize}

\InputFile
Il file \inputfile{} è composto da M+2 righe. La prima riga contiene i due interi $N$, $M$. La seconda riga contiene gli $N$ interi $B_i$ separati da uno spazio. Le successive $M$ righe contengono ciascuna i due interi \texttt{da[$i$]} e \texttt{a[$i$]}.

\OutputFile
Il file \outputfile{} è composto da un'unica riga contenente un unico intero, la risposta a questo problema.

\pagebreak
% Assunzioni
\Constraints
\begin{itemize}[nolistsep, itemsep=2mm]
	\item $2 \le N \le 10\,000$.
	\item $1 \le M \le 100\,000$.
	\item $0 \le B_i \le 1$ per ogni $i=0\ldots N-1$.
	\item $0 \le \texttt{da}_i, \texttt{a}_i \le N-1$ per ogni $i=0\ldots M-1$.
	\item Ci possono essere pi\`u tiri di carrucole tra la stessa coppia di alberi, e anche tiri di carrucole all'interno dello stesso albero.
\end{itemize}

\Scoring
Il tuo programma verrà testato su diversi test case raggruppati in subtask.
Per ottenere il punteggio relativo ad un subtask, è necessario risolvere
correttamente tutti i test relativi ad esso.

\begin{itemize}[nolistsep,itemsep=2mm]
  \item \textbf{\makebox[2cm][l]{Subtask 1} [10 punti]}: Casi d'esempio.
  \item \textbf{\makebox[2cm][l]{Subtask 2} [20 punti]}: $N, M \leq 50$.
  \item \textbf{\makebox[2cm][l]{Subtask 3} [40 punti]}: $N, M \leq 1000$.
  \item \textbf{\makebox[2cm][l]{Subtask 4} [30 punti]}: Nessuna limitazione specifica.
\end{itemize}

% Esempi


\Examples
\begin{example}
\exmpfile{carrucole.input0.txt}{carrucole.output0.txt}%
\exmpfile{carrucole.input1.txt}{carrucole.output1.txt}%
\end{example}


\Explanation
Nel \textbf{primo caso di esempio}, rimangono tre tiri di carrucola: tra gli alberi 1--2, 2--3 e 1--3.\\[2mm]
Nel \textbf{secondo caso di esempio}, l'unico tiro di carrucola non \`e stato abbattuto.
