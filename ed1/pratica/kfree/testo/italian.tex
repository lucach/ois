% \usepackage{xcolor}
% \usepackage{afterpage}
\usepackage{pifont,mdframed}
% \usepackage[bottom]{footmisc}

\makeatletter
\gdef\this@inputfilename{input.txt}
\gdef\this@outputfilename{output.txt}
\makeatother

\createsection{\Grader}{Grader di prova}

\newcommand{\inputfile}{\texttt{input.txt}}
\newcommand{\outputfile}{\texttt{output.txt}}

\newenvironment{warning}
  {\par\begin{mdframed}[linewidth=2pt,linecolor=gray]%
    \begin{list}{}{\leftmargin=1cm
                   \labelwidth=\leftmargin}\item[\Large\ding{43}]}
  {\end{list}\end{mdframed}\par}


Il professore di informatica di Gabriele, noto scienziato delle merendine, è convinto che tutti i problemi NP-completi possano essere ridotti al problema del massimo sottinsieme $K$-free. Dato un numero intero positivo $K$, un insieme $A$ di numeri interi positivi è detto $K$-free se rispetta la seguente condizione:
\[ K\cdot a \not\in A\quad \text{per ogni } a \in A. \]
In altre parole, se un insieme $K$-free contiene $a$, non può contenere anche $K \cdot a$.

Diventa a questo punto cruciale determinare la dimensione del più grande sottoinsieme di un insieme dato, che ha la proprietà di essere $K-$free\footnote{Notare che la proprietà di un sottoinsieme di essere $K$-free è una proprietà solo del sottoinsieme, che non dipende in alcun modo dall'insieme di partenza.}.


\InputFile
La prima riga del file di input contiene gli interi $N$ e $K$, il numero di elementi nell'insieme iniziale e il valore di $K$. La seconda riga contiene gli $N$ interi distinti $a_1, \ldots, a_N$ dell'insieme. 

\OutputFile
In output, stampare la dimensione del più grande sottoinsieme $K-$free dell'insieme dato in input.

\Implementation
Dovrai sottoporre esattamente un file con estensione \texttt{.c}, \texttt{.cpp} o \texttt{.pas}.

\begin{warning}
Tra gli allegati a questo task troverai un template (\texttt{kfree.c}, \texttt{kfree.cpp}, \texttt{kfree.pas}) con un esempio di implementazione da completare.
\end{warning}

Se sceglierai di utilizzare il template, dovrai implementare la seguente funzione:
\begin{center}\begin{tabularx}{\textwidth}{|c|X|}
\hline
C/C++  & \verb|int trova(int N, int K, int insieme[]);|\\
\hline
Pascal & \verb|function trova(N, K: longint; var insieme: array of longint): longint;|\\
\hline
\end{tabularx}\end{center}
La funzione riceverà come parametri i valori $N$ e $K$ e un array di interi che rappresenta l'insieme iniziale e dovrà ritornare la risposta al problema, che verrà stampata sul file di output.

\Constraints 
\begin{itemize}[nolistsep,itemsep=2mm]
  \item $1 \le N \le 100\,000$.
  \item $1 \le K \le 1000$.
  \item $1 \le a_i \le 100\,000$, per ogni $i=1,\ldots,N$.
  \item Gli interi $a_1, \ldots, a_N$ sono distinti.
\end{itemize}

\pagebreak
\Scoring
Il tuo programma verrà testato su diversi test case raggruppati in subtask.
Per ottenere il punteggio relativo ad un subtask, è necessario risolvere
correttamente tutti i test relativi ad esso.

\begin{itemize}[nolistsep,itemsep=2mm]
  \item \textbf{\makebox[2cm][l]{Subtask 1} [10 punti]}: Caso d'esempio.
  \item \textbf{\makebox[2cm][l]{Subtask 2} [20 punti]}: $N \le 100, K = 1$.
  \item \textbf{\makebox[2cm][l]{Subtask 3} [40 punti]}: $N \le 500, K \le 100$.
  \item \textbf{\makebox[2cm][l]{Subtask 4} [30 punti]}: Nessuna limitazione specifica.
\end{itemize}

\Examples
\begin{example}
\exmp{
6 2
2 3 6 5 4 10
}{%
3
}%
\end{example}

\Explanation
Nel caso di esempio un sottoinsieme $2-$free di dimensione massima è $\{4, 5, 6\}$, infatti non contiene né 8, né 10 né 12. Esistono altri sottoinsiemi $2-$free della stessa dimensione, come ad esempio $\{2, 3, 5\}$ o  $\{2, 3, 10\}$, mentre altri sottoinsiemi della stessa dimensione non sono $2-$free (come ad esempio $\{2, 3, 4\}$).
