% \usepackage{xcolor}
% \usepackage{afterpage}
\usepackage{pifont,mdframed}
% \usepackage[bottom]{footmisc}

\makeatletter
\gdef\this@inputfilename{input.txt}
\gdef\this@outputfilename{output.txt}
\makeatother

\newcommand{\inputfile}{\texttt{input.txt}}
\newcommand{\outputfile}{\texttt{output.txt}}

\newenvironment{warning}
  {\par\begin{mdframed}[linewidth=2pt,linecolor=gray]%
    \begin{list}{}{\leftmargin=1cm
                   \labelwidth=\leftmargin}\item[\Large\ding{43}]}
  {\end{list}\end{mdframed}\par}

Gabriele stava fotocopiando un fascicolo di $N$ pagine, quando arrivato a dover fotocopiare l'ultima pagina una folata di vento ha sollevato tutti i fogli, mescolandoli. Ora sul pavimento della camera di Gabriele sono sparpagliati $2N - 1$ fogli, e Gabriele non sa quale era la pagina che mancava ancora da fotocopiare.

Data la lista dei numeri di pagina dei fogli sparpagliati sul pavimento, aiuta Gabriele a capire qual è l'unico numero di pagina che è presente solo una volta.

\Implementation
Dovrai sottoporre esattamente un file con estensione \texttt{.c}, \texttt{.cpp} o \texttt{.pas}.

\begin{warning}
Tra gli allegati a questo task troverai un template (\texttt{duplicato.c}, \texttt{duplicato.cpp}, \texttt{duplicato.pas}) con un esempio di implementazione da completare.
\end{warning}

Se sceglierai di utilizzare il template, dovrai implementare la seguente funzione:
\begin{center}\begin{tabularx}{\textwidth}{|c|X|}
\hline
C/C++  & \verb|int trova(int N, int P[]);|\\
\hline
Pascal & \verb|function trova(N: longint; var P: array of longint): longint;|\\
\hline
\end{tabularx}\end{center}
In cui:
\begin{itemize}[nolistsep]
  \item L'intero $N$ rappresenta il numero di pagine del fascicolo
  \item L'array \texttt{P}, indicizzato da $0$ a $2N-2$, contiene i numeri di pagina dei fogli sparpagliati sul pavimento
  \item La funzione dovrà restituire il numero di pagina del foglio che Gabriele deve ancora fotocopiare, che verrà stampato sul file di output.
\end{itemize}

\InputFile
Il file \inputfile{} è composto da due righe. La prima riga contiene l'unico intero $N$. La seconda riga contiene i $2N - 1$ interi $P_i$ separati da uno spazio.

\OutputFile
Il file \outputfile{} è composto da un'unica riga contenente un unico intero, la risposta a questo problema.

% Assunzioni
\Constraints
\begin{itemize}[nolistsep, itemsep=2mm]
	\item $2 \le N \le 100\,000$.
	\item $1 \le P_i \le 1\,000\,000\,000$ per ogni $i=0\ldots 2N-2$.
	\item Ogni numero di pagina compare esattamente 2 volte, ad eccezione di un numero di pagina che appare esattamente 1 volta, e che coincide con la risposta al problema.
\end{itemize}

\Scoring
Il tuo programma verrà testato su diversi test case raggruppati in subtask.
Per ottenere il punteggio relativo ad un subtask, è necessario risolvere
correttamente tutti i test relativi ad esso.

\begin{itemize}[nolistsep,itemsep=2mm]
  \item \textbf{\makebox[2cm][l]{Subtask 1} [10 punti]}: Casi d'esempio.
  \item \textbf{\makebox[2cm][l]{Subtask 2} [20 punti]}: $N \leq 10$; inoltre, i numeri di pagina non superano il valore $100\,000$.
  \item \textbf{\makebox[2cm][l]{Subtask 3} [40 punti]}: $N \leq 1\,000$; inoltre, i numeri di pagina non superano il valore $100\,000$.
  \item \textbf{\makebox[2cm][l]{Subtask 4} [30 punti]}: Nessuna limitazione specifica.
\end{itemize}

% Esempi
\Examples
\begin{example}
\exmp{
2
1 1 2
}{%
2
}%
\end{example}
\begin{example}
\exmp{
5
4 5 4 10 8 10 3 5 8
}{%
3
}%
\end{example}


\Explanation
Nel \textbf{primo caso di esempio}, l'unico numero non ripetuto è 2.\\[2mm]
Nel \textbf{secondo caso di esempio}, l'unico numero non ripetuto è 3.
