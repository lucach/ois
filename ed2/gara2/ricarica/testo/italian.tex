\usepackage{xcolor}
\usepackage{afterpage}
\usepackage{pifont,mdframed}
\usepackage[bottom]{footmisc}

\makeatletter
\gdef\this@inputfilename{input.txt}
\gdef\this@outputfilename{output.txt}
\makeatother

\newcommand{\inputfile}{\texttt{input.txt}}
\newcommand{\outputfile}{\texttt{output.txt}}

\newenvironment{warning}
  {\par\begin{mdframed}[linewidth=2pt,linecolor=gray]%
    \begin{list}{}{\leftmargin=1cm
                   \labelwidth=\leftmargin}\item[\Large\ding{43}]}
  {\end{list}\end{mdframed}\par}

	Il cellulare di Giorgio ha la sua età e ormai si scarica molto velocemente. Per ogni minuto di utilizzo perde un punto di carica, proprio come per ogni minuto in cui è collegato alla corrente ne acquista uno. Se poi raggiunge zero punti di carica, va in \emph{crash} e diventa necessario portarlo all'assistenza tecnica.

	Giorgio deve ora uscire di casa per una serie di commissioni di una durata complessiva di $M$ minuti, numerati da 1 a $M$. Per evitare di rimanere con il cellulare scarico, ha quindi stilato un elenco di tutti gli $N$ intervalli di minuti $(A_i, B_i)$ in cui sa che potrà collegarlo alla corrente e recuperare punti di carica. In tutti gli altri minuti, il cellulare perderà punti di carica.

	Con quanti punti di carica Giorgio deve uscire di casa al minimo, per essere certo di non rimanere mai con il cellulare scarico in nessuno degli $M$ minuti?

\Implementation
Dovrai sottoporre esattamente un file con estensione \texttt{.c}, \texttt{.cpp} o \texttt{.pas}.

\begin{warning}
Tra gli allegati a questo task troverai un template (\texttt{ricarica.c}, \texttt{ricarica.cpp}, \texttt{ricarica.pas}) con un esempio di implementazione da completare.
\end{warning}

Se sceglierai di utilizzare il template, dovrai implementare la seguente funzione:
\begin{center}\begin{tabularx}{\textwidth}{|c|X|}
\hline
C/C++  & \verb|int ricarica(int N, int M, int A[], int B[]);|\\
\hline
Pascal & \verb|function ricarica(N, M: longint; var A, B: array of longint): longint;|\\
\hline
\end{tabularx}\end{center}
In cui:
\begin{itemize}[nolistsep]
  \item L'intero $N$ rappresenta il numero di intervalli in cui Giorgio collegherà il cellulare alla corrente.
  \item L'intero $M$ rappresenta il numero di minuti totali delle commissioni.
  \item Gli array \texttt{A} e \texttt{B}, indicizzati da $0$ a $N-1$, contengono gli intervalli di minuti $(A_i, B_i)$ in cui il cellulare sarà collegato alla corrente.
  \item La funzione dovrà restituire la minima quantità di punti di carica necessari affinché il cellulare non si scarichi durante gli $M$ minuti, che verrà stampata sul file di output.
\end{itemize}

\InputFile
Il file \inputfile{} è composto da $N+1$ righe. La prima riga contiene i due interi $N$, $M$. La riga $i+1$ per $i=1 \ldots N$ contiene i due interi $A_i$, $B_i$ separati da uno spazio.

\OutputFile
Il file \outputfile{} è composto da un'unica riga contenente un unico intero, la risposta a questo problema.

\pagebreak
% Assunzioni
\Constraints
\begin{itemize}[nolistsep, itemsep=2mm]
	\item $1 \le N \le 100\,000$.
	\item $1 \le M \le 2\,000\,000\,000$.
	\item $1 \le A_i \le B_i \le M$ per ogni $i=0\ldots N-1$.
	\item $B_i < A_{i+1}$ per ogni $i=0\ldots N-2$.
\end{itemize}

\Scoring
Il tuo programma verrà testato su diversi test case raggruppati in subtask.
Per ottenere il punteggio relativo ad un subtask, è necessario risolvere
correttamente tutti i test relativi ad esso.

\begin{itemize}[nolistsep,itemsep=2mm]
  \item \textbf{\makebox[2cm][l]{Subtask 1} [10 punti]}: Casi d'esempio.
  \item \textbf{\makebox[2cm][l]{Subtask 2} [20 punti]}: $N \leq 10$.
  \item \textbf{\makebox[2cm][l]{Subtask 3} [40 punti]}: $M \leq 10000$.
  \item \textbf{\makebox[2cm][l]{Subtask 4} [30 punti]}: Nessuna limitazione specifica.
\end{itemize}

% Esempi


\Examples
\begin{example}
\exmpfile{ricarica.input0.txt}{ricarica.output0.txt}%
\exmpfile{ricarica.input1.txt}{ricarica.output1.txt}%
\end{example}


\Explanation
Nel \textbf{primo caso di esempio}, la carica del cellulare segue questo schema:
\begin{center}
\begin{tabular}{ccccccccccccccc}
	\hline
	\multicolumn{1}{|c|}{$+$} & \multicolumn{1}{c|}{$-$} & \multicolumn{1}{c|}{$-$} & \multicolumn{1}{c|}{$-$} & \multicolumn{1}{c|}{$+$} & \multicolumn{1}{c|}{$+$} & \multicolumn{1}{c|}{$-$} & \multicolumn{1}{c|}{$+$} & \multicolumn{1}{c|}{$+$} & \multicolumn{1}{c|}{$-$} & \multicolumn{1}{c|}{$-$} & \multicolumn{1}{c|}{$+$} & \multicolumn{1}{c|}{$+$} & \multicolumn{1}{c|}{$+$} & \multicolumn{1}{c|}{$-$} \\
	\hline
	1 & 2 & 3 & 4 & 5 & 6 & 7 & 8 & 9 & 10 & 11 & 12 & 13 & 14 & 15 \\
\end{tabular}
\end{center}
La carica minore viene quindi raggiunta alla fine del quarto minuto, in cui i punti residui saranno $3 + 1 - 1 - 1 - 1 = 1$ e pertanto 3 è proprio la carica minima necessaria.\\[2mm]
Nel \textbf{secondo caso di esempio}, se il cellulare parte con 15 punti di carica si ritrova con 9 all'inizio del settimo minuto, e poi con 19 alla fine del sedicesimo, con 8 all'inizio del ventottesimo, con 41 alla fine del sessantesimo e infine con 1 alla fine del centesimo.
