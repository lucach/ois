\usepackage{xcolor}
\usepackage{afterpage}
\usepackage{pifont,mdframed}
\usepackage[bottom]{footmisc}

\makeatletter
\gdef\this@inputfilename{input.txt}
\gdef\this@outputfilename{output.txt}
\makeatother

\newcommand{\inputfile}{\texttt{input.txt}}
\newcommand{\outputfile}{\texttt{output.txt}}

\newenvironment{warning}
  {\par\begin{mdframed}[linewidth=2pt,linecolor=gray]%
    \begin{list}{}{\leftmargin=1cm
                   \labelwidth=\leftmargin}\item[\Large\ding{43}]}
  {\end{list}\end{mdframed}\par}

	La \emph{OIS S.p.A.} vuole aprire $F$ nuove filiali scelte ciascuna tra $N$ possibili citt\`a. Le $N$ citt\`a considerate sono tutte disposte lungo l'\emph{Autostrada del Sole}, ciascuna a un diverso chilometro $K_i$ per $i=0, \ldots, N$. Data una possibile scelta delle filiali, definiamo il suo \emph{bilanciamento} come la minima distanza tra due qualsiasi delle filiali scelte (la distanza tra due filiali \`e pari alla differenza di chilometraggio delle relative citt\`a).

	Qual \`e il massimo bilanciamento ottenibile per una scelta di $F$ filiali tra le $N$ citt\`a date?


\Implementation
Dovrai sottoporre esattamente un file con estensione \texttt{.c}, \texttt{.cpp} o \texttt{.pas}.

\begin{warning}
Tra gli allegati a questo task troverai un template (\texttt{filiali.c}, \texttt{filiali.cpp}, \texttt{filiali.pas}) con un esempio di implementazione da completare.
\end{warning}

Se sceglierai di utilizzare il template, dovrai implementare la seguente funzione:
\begin{center}\begin{tabularx}{\textwidth}{|c|X|}
\hline
C/C++  & \verb|int apri(int N, int F, int K[]);|\\
\hline
Pascal & \verb|function apri(N, F: longint; var K: array of longint): longint;|\\
\hline
\end{tabularx}\end{center}
In cui:
\begin{itemize}[nolistsep]
  \item L'intero $N$ rappresenta il numero di citt\`a possibili per le filiali.
  \item L'intero $F$ rappresenta il numero di filiali da aprire.
  \item L'array \texttt{K}, indicizzato da $0$ a $N-1$, contiene il chilometro corrispondente a ciascuna citt\`a.
  \item La funzione dovrà restituire il miglior bilanciamento per $F$ filiali, che verrà stampato sul file di output.
\end{itemize}

\InputFile
Il file \inputfile{} è composto da due righe. La prima riga contiene i due interi $N$ ed $F$. La seconda riga contiene gli $N$ interi $K_i$ separati da uno spazio.

\OutputFile
Il file \outputfile{} è composto da un'unica riga contenente un unico intero, la risposta a questo problema.

% Assunzioni
\Constraints
\begin{itemize}[nolistsep, itemsep=2mm]
	\item $2 \le F \le 1000$.
	\item $1 \le N \le 1\,000\,000$.
	\item $0 \le K_i \le K_{i+1} < 2^{31}$ per ogni $i=0\ldots N-1$.
\end{itemize}

\Scoring
Il tuo programma verrà testato su diversi test case raggruppati in subtask.
Per ottenere il punteggio relativo ad un subtask, è necessario risolvere
correttamente tutti i test relativi ad esso.

\begin{itemize}[nolistsep,itemsep=2mm]
  \item \textbf{\makebox[2cm][l]{Subtask 1} [10 punti]}: Casi d'esempio.
  \item \textbf{\makebox[2cm][l]{Subtask 2} [10 punti]}: $F = 3$.
  \item \textbf{\makebox[2cm][l]{Subtask 3} [10 punti]}: $F = 4$.
  \item \textbf{\makebox[2cm][l]{Subtask 4} [10 punti]}: $F \le 7$.
  \item \textbf{\makebox[2cm][l]{Subtask 5} [20 punti]}: $N, F \leq 100$.
  \item \textbf{\makebox[2cm][l]{Subtask 6} [20 punti]}: $N \leq 5000$.
  \item \textbf{\makebox[2cm][l]{Subtask 7} [20 punti]}: Nessuna limitazione specifica.
\end{itemize}

% Esempi


\Examples
\begin{example}
\exmpfile{filiali.input0.txt}{filiali.output0.txt}%
\exmpfile{filiali.input1.txt}{filiali.output1.txt}%
\end{example}


\Explanation
Nel \textbf{primo caso di esempio}, conviene aprire le filiali nella prima e nell'ultima citt\`a.\\[2mm]
Nel \textbf{secondo caso di esempio}, due filiali vengono aperte nella prima e nell'ultima citt\`a e la rimanente filiale pu\`o essere aperta indifferentemente nella seconda o nella terza citt\`a.
