% \usepackage{xcolor}
% \usepackage{afterpage}
\usepackage{pifont,mdframed}
% \usepackage[bottom]{footmisc}

\makeatletter
\gdef\this@inputfilename{input.txt}
\gdef\this@outputfilename{output.txt}
\makeatother

\newcommand{\inputfile}{\texttt{input.txt}}
\newcommand{\outputfile}{\texttt{output.txt}}

\newenvironment{warning}
  {\par\begin{mdframed}[linewidth=2pt,linecolor=gray]%
    \begin{list}{}{\leftmargin=1cm
                   \labelwidth=\leftmargin}\item[\Large\ding{43}]}
  {\end{list}\end{mdframed}\par}

La \emph{SteamPower S.P.A.} è di nuovo alla ricerca di una soluzione rapida ed efficace per far quadrare i conti. Per risolvere la questione, il consiglio di amministrazione ha deciso di nascondere gli illeciti della società tramite la costruzione di un lunghissimo centro sportivo destinato allo spettacolo dei trampolini di Giorgio. Con una sola mossa, in altre parole, la società riuscirà a celare le irregolarità nei bilanci e farsi una grossa pubblicità, e tutti i dirigenti sono entusiasti (già si parla di un torneo di rugby...). 

Giorgio è stato tassativo sui requisiti della costruzione: questa dovrà essere lunga esattamente $K$ metri, non uno di meno e non uno di più. Per accomodare la richiesta la società si è affidata al proprio consulente di fiducia, Gabriele.

Gabriele ha individuato la via più lunga della città, e sta valutando in che punto esatto posizionare la struttura per risparmiare il più possibile. Infatti, la via è completamente edificata e per poter iniziare a costruire il palazzo è necessario prima demolire le costruizioni già esistenti sul terreno, pagando un costo proporzionale al loro valore.

La strada è lunga $N$ metri, e per ogni metro di edifici Gabriele ne ha stimato il valore $V_i$. Ora deve scegliere $K$ metri consecutivi di strada, in modo da minimizzare la somma dei valori degli edifici da demolire per fare spazio al centro sportivo.

\Implementation
Dovrai sottoporre esattamente un file con estensione \texttt{.c}, \texttt{.cpp} o \texttt{.pas}.

\begin{warning}
Tra gli allegati a questo task troverai un template (\texttt{costruzioni.c}, \texttt{costruzioni.cpp}, \texttt{costruzioni.pas}) con un esempio di implementazione da completare.
\end{warning}

Se sceglierai di utilizzare il template, dovrai implementare la seguente funzione:
\begin{center}\begin{tabularx}{\textwidth}{|c|X|}
\hline
C/C++  & \verb|int demolisci(int N, int K, int V[]);|\\
\hline
Pascal & \verb|function demolisci(N, K: longint; var V: array of longint): longint;|\\
\hline
\end{tabularx}\end{center}
In cui:
\begin{itemize}[nolistsep]
  \item L'intero $N$ rappresenta la lunghezza, in metri, della strada.
  \item L'intero $K$ rappresenta la lunghezza, in metri, del centro sportivo da costruire.
  \item L'array \texttt{V}, indicizzato da $0$ a $N-1$, contiene il valore degli edifici presenti nel metro $i$-esimo di strada.
  \item La funzione dovrà restituire la minima somma di valori di un segmento di esattamente $K$ metri consecutivi di strada, che verrà stampato sul file di output.
\end{itemize}

\InputFile
Il file \inputfile{} è composto da due righe. La prima riga contiene gli interi $N$ e $K$. La seconda riga contiene gli $N$ interi $V_i$ separati da uno spazio.

\OutputFile
Il file \outputfile{} è composto da un'unica riga contenente un unico intero, la risposta a questo problema.

% Assunzioni
\Constraints
\begin{itemize}[nolistsep, itemsep=2mm]
	\item $1 \le K \le N \le 100\,000$.
	\item $1 \le V_i \le 10\,000$ per ogni $i=0\ldots N-1$.
\end{itemize}

\Scoring
Il tuo programma verrà testato su diversi test case raggruppati in subtask.
Per ottenere il punteggio relativo ad un subtask, è necessario risolvere
correttamente tutti i test relativi ad esso.

\begin{itemize}[nolistsep,itemsep=2mm]
  \item \textbf{\makebox[2cm][l]{Subtask 1} [10 punti]}: Casi d'esempio.
  \item \textbf{\makebox[2cm][l]{Subtask 2} [20 punti]}: $N \leq 100$.
  \item \textbf{\makebox[2cm][l]{Subtask 3} [40 punti]}: $N \leq 10\,000$, $K \le 50$.
  \item \textbf{\makebox[2cm][l]{Subtask 4} [30 punti]}: Nessuna limitazione specifica.
\end{itemize}

% Esempi
\Examples
\begin{example}
\exmp{
5 3
5 7 6 4 8
}{%
17
}%
\end{example}
\begin{example}
\exmp{
9 7
1 2 3 4 5 4 3 1 1
}{%
21
}%
\end{example}


\Explanation
Nel \textbf{primo caso di esempio} conviene demolire gli edifici di valore $7, 6, 4$.\\[2mm]
Nel \textbf{secondo caso di esempio} conviene demolire gli ultimi 7 edifici.
