% \usepackage{xcolor}
% \usepackage{afterpage}
\usepackage{pifont,mdframed}
% \usepackage[bottom]{footmisc}

\makeatletter
\gdef\this@inputfilename{input.txt}
\gdef\this@outputfilename{output.txt}
\makeatother

\newcommand{\inputfile}{\texttt{input.txt}}
\newcommand{\outputfile}{\texttt{output.txt}}

\newenvironment{warning}
  {\par\begin{mdframed}[linewidth=2pt,linecolor=gray]%
    \begin{list}{}{\leftmargin=1cm
                   \labelwidth=\leftmargin}\item[\Large\ding{43}]}
  {\end{list}\end{mdframed}\par}

	Giorgio si è appassionato alla crittografia, e dopo aver studiato con attenzione tutti i più importanti algoritmi di crittografia simmetrica, ha pensato bene di introdurre un nuovo tipo di crittografia: la \emph{crittografia riflessa}.
	
	Più precisamente, per codificare i suoi messaggi Giorgio intende utilizzare una funzione che dato un numero $N$, ne calcola il numero speculare (con tutte le cifre invertite dall'ultima alla prima) e lo somma al numero di partenza. Aiuta Giorgio a scrivere la funzione di codifica!

\Implementation
Dovrai sottoporre esattamente un file con estensione \texttt{.c}, \texttt{.cpp} o \texttt{.pas}.

\begin{warning}
Tra gli allegati a questo task troverai un template (\texttt{crittografia.c}, \texttt{crittografia.cpp}, \texttt{crittografia.pas}) con un esempio di implementazione da completare.
\end{warning}

Se sceglierai di utilizzare il template, dovrai implementare la seguente funzione:
\begin{center}\begin{tabularx}{\textwidth}{|c|X|}
\hline
C/C++  & \verb|int codifica(int N);|\\
\hline
Pascal & \verb|function codifica(N: longint): longint;|\\
\hline
\end{tabularx}\end{center}
In cui:
\begin{itemize}[nolistsep]
  \item L'intero $N$ rappresenta il numero da codificare.
  \item La funzione dovrà restituire la somma tra il numero $N$ e il suo riflesso, che verrà stampata sul file di output.
\end{itemize}

\InputFile
Il file \inputfile{} è composto da un'unica riga contenente l'unico intero $N$.

\OutputFile
Il file \outputfile{} è composto da un'unica riga contenente un unico intero, la risposta a questo problema.

% Assunzioni
\Constraints
\begin{itemize}[nolistsep, itemsep=2mm]
	\item $1 \le N \le 1\,000\,000\,000$.
\end{itemize}

\pagebreak
\Scoring
Il tuo programma verrà testato su diversi test case raggruppati in subtask.
Per ottenere il punteggio relativo ad un subtask, è necessario risolvere
correttamente tutti i test relativi ad esso.

\begin{itemize}[nolistsep,itemsep=2mm]
  \item \textbf{\makebox[2cm][l]{Subtask 1} [10 punti]}: Casi d'esempio.
  \item \textbf{\makebox[2cm][l]{Subtask 2} [20 punti]}: $N \leq 99$.
  \item \textbf{\makebox[2cm][l]{Subtask 3} [40 punti]}: Tutte le cifre di $N$ sono $0$ oppure $1$.
  \item \textbf{\makebox[2cm][l]{Subtask 4} [30 punti]}: Nessuna limitazione specifica.
\end{itemize}

% Esempi
\Examples
\begin{example}
\exmp{
1023
}{%
4224
}%
\end{example}
\begin{example}
\exmp{
56390
}{%
65755
}%
\end{example}


\Explanation
Nel \textbf{primo caso di esempio}, $1023 + 3201$ fa $4224$.\\[2mm]
Nel \textbf{secondo caso di esempio}, $56\,390 + 09\,365$ fa $65\,755$.
