% \usepackage{xcolor}
% \usepackage{afterpage}
\usepackage{pifont,mdframed}
% \usepackage[bottom]{footmisc}

\makeatletter
\gdef\this@inputfilename{input.txt}
\gdef\this@outputfilename{output.txt}
\makeatother

\newcommand{\inputfile}{\texttt{input.txt}}
\newcommand{\outputfile}{\texttt{output.txt}}

\newenvironment{warning}
  {\par\begin{mdframed}[linewidth=2pt,linecolor=gray]%
    \begin{list}{}{\leftmargin=1cm
                   \labelwidth=\leftmargin}\item[\Large\ding{43}]}
  {\end{list}\end{mdframed}\par}

	La \emph{SteamPower S.P.A.}, azienda leader mondiale nel campo delle macchine a vapore portatili, non accenna ad uscire dal periodo di crisi nonostante i forti e ben ponderati tagli al personale di recente effettuati. Per fortuna il \emph{CEO} ha avuto una nuova geniale idea: affidarsi al massimo esperto mondiale in campo di finanza creativa. L'esperto ha già ricevuto il bilancio dal reparto contabilità, e la situazione è disastrosa: arrivati a questo punto l'unico modo che conosce per rassicurare gli azionisti e prendere tempo è quello di effettuare qualche piccolo ritocco ai libri contabili.

	Più precisamente, vuole ricorrere al suo fedele bianchetto per correggere il totale $U$ delle uscite. Inoltre, grazie agli insegnamenti di lunghi anni di esperienza, sa che per contenere i rischi dell'operazione non deve assolutamente eccedere le $K$ cifre cancellate (sulle $N$ complessive del totale $U$).

	Aiuta l'esperto di finanza creativa a trovare il minimo intero ottenibile cancellando $K$ cifre dall'intero $U$!

\Implementation
Dovrai sottoporre esattamente un file con estensione \texttt{.c}, \texttt{.cpp} o \texttt{.pas}.

\begin{warning}
Tra gli allegati a questo task troverai un template (\texttt{bilancio.c}, \texttt{bilancio.cpp}, \texttt{bilancio.pas}) con un esempio di implementazione da completare.
\end{warning}

Se sceglierai di utilizzare il template, dovrai implementare la seguente funzione:
\begin{center}\begin{tabularx}{\textwidth}{|c|X|}
\hline
C/C++  & \verb|void bianchetta(int N, int K, int U[], int C[]);|\\
\hline
Pascal & \verb|procedure bianchetta(N, K: longint; var U, C: array of longint);|\\
\hline
\end{tabularx}\end{center}
In cui:
\begin{itemize}[nolistsep]
  \item L'intero $N$ rappresenta il numero di cifre del totale delle uscite.
  \item L'intero $K$ rappresenta il numero di cifre da cancellare.
  \item L'array $U$, indicizzato da $0$ a $N-1$, contiene l'elenco delle cifre del totale delle uscite, con la cifra più significativa in posizione $0$ e così via fino alla cifra delle unità in posizione $N-1$.
  \item La funzione dovrà riempire l'array $C$, indicizzato da $0$ a $N-K-1$, con l'elenco delle cifre rimaste dopo la cancellazione (dalla più alla meno significativa), che verrà stampato sul file di output.
\end{itemize}

\InputFile
Il file \inputfile{} è composto da due righe. La prima riga contiene i due interi $N$ e $K$. La seconda riga contiene le $N$ cifre $U_i$ del totale delle uscite $U$, separate da uno spazio.

\OutputFile
Il file \outputfile{} è composto da un'unica riga contenente $N-K$ cifre separate da uno spazio, la risposta a questo problema.

\pagebreak
% Assunzioni
\Constraints
\begin{itemize}[nolistsep, itemsep=2mm]
	\item $1 \le K < N \le 1\,000\,000$.
	\item $0 \le U_i \le 9$ per ogni $i = 0 \ldots N-1$, e $U_0 \ge 1$.
\end{itemize}

\Scoring
Il tuo programma verrà testato su diversi test case raggruppati in subtask.
Per ottenere il punteggio relativo ad un subtask, è necessario risolvere
correttamente tutti i test relativi ad esso.

\begin{itemize}[nolistsep,itemsep=2mm]
  \item \textbf{\makebox[2cm][l]{Subtask 1} [10 punti]}: Casi d'esempio.
  \item \textbf{\makebox[2cm][l]{Subtask 2} [20 punti]}: $N \leq 10$.
  \item \textbf{\makebox[2cm][l]{Subtask 3} [30 punti]}: $N \leq 200$.
  \item \textbf{\makebox[2cm][l]{Subtask 3} [20 punti]}: $N \leq 10\,000$.
  \item \textbf{\makebox[2cm][l]{Subtask 4} [20 punti]}: Nessuna limitazione specifica.
\end{itemize}

% Esempi
\Examples
\begin{example}
\exmp{
5 2
1 9 5 0 3
}{%
1 0 3
}%
\end{example}
\begin{example}
\exmp{
9 3
9 4 1 5 7 1 1 2 3
}{%
1 5 1 1 2 3
}%
\end{example}
\begin{example}
\exmp{
10 5
1 3 2 0 2 1 8 5 3 6
}{%
0 1 5 3 6
}%
\end{example}


\Explanation
Nel \textbf{primo caso di esempio} conviene cancellare le due cifre più alte (5 e 9).\\[2mm]
Nel \textbf{secondo caso di esempio} conviene cancellare le due cifre più significative (9 e 4), e la restante cifra più alta (il 7).\\[2mm]
Nel \textbf{terzo caso di esempio} conviene cancellare le tre cifre più significative (1, 3 e 2), il 2 ancora successivo e la restante cifra più alta (l'8).
